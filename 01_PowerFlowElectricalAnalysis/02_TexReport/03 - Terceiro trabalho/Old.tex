%DO NOT INCLUDE THIS FILE
%COMMENTS AND TODOS

%Cap4
%pag 3 equação errada pois algumas das variáveis já sabemos. 
%Objetivo -> Determinar V, Theta
%NPQ -> Barra de carga
%NPV -> Barra de geração
%Numero de variaveis = Numero de incognitas e equações LI
%Gkm condutâncias
%Bkm aceptancias 
%que vem da matriz de admitancia
%Pag 8
%
%1 barra de referencia
%PQ barra de carga
%PV barra com geração
%Angulos conhecidos -> só na barra de referencia - Tetha: 2,3,4
%tensões nas barras V: 2 , 4
%Pag 11
%v - numero de iterações
%Pag 15
%equações das matrizes jacobianas
%Observar a dimensão das matrizes
%H e L são quadradas e sao simetricas - se existe H23 existe H32, mas não são necessariamente iguais
%Pag 17 Exemplo 
%Matriz jacobiana
%Pag 18
%Todas as injeções de P ativa e reativa - todos os angulos e todas as tensõe
%linhas
%1 derivasd p1
%2 derivada p2
%3 derivada p3
%4 derivada q1
%5 derivada q2
%6 derivada q3
%colunas 
%...
%elemento 3,4 -> diz respeito derivadad de p3 com relação a v1
%linha g(x) 
%Coluna delta (x)
%Pag 19
%Para J^-1 , onde tem infinito vai a 0
%
%Critérios de convergência
%24/apr/2020
%Matlab
%loadflow_v3.m - P2GTD.m
%potencia em kva
%lendo o numero de colunas da rede
%nb,colunas size
%nr, colunas size

%ler info das barras
%for k = 1:nb
%*tudo dividido pela potencia de base pra transformar em PU

%pnom = potencia gerada - consumida
%numint(barras(k,1)) = k; bobeira

%Lendo info dos ramos
%for L = 1:nr
%shunt bsh1 dividido por 2
%ler posição do tap

%montagem da matriz admitancia

%cria uma matriz de zeros, Y (nr, nb)

%Shunts de barra - Verifica em todas as barras onde tem caps, na diagonal Y(k,k) = i*bshk(k)

%Varredura pelos ramos
%calcula admitancia y
%Equações são as da matriz de admitancia no slide (??)
%Chutar valores 
%se for do tipo v theta, 
%se for pv, chute valor para teta
%pq, chute v e teta
%chute flat, angulo 0 e v 1
%Calculo das potencias nodais
%para cada barra tem um p calc e um q calc
%varredura pelos ramos
%***Definição de injeção de potencia: é o somatório dos fluxos dos ramos

%...

%Estar atentos as dimensões das matrizes
%Calculo das injeções de potencia e matriz jacobiana - numero de ramos não de barras



%para o trabalho descoplado pagina 40

%pode ser o desacoplado rápido, ou o normal
