\chapter{C\'odigo comentado}
\label{SectionCodigo}

Os códigos fonte desse trabalho bem como histórico de modificação, podem ser encontrados no endereço: \textit{\href{https://github.com/phneves/ElectricPowerSystemsAnalysisTools}{https://github.com/phneves/ElectricPowerSystemsAnalysisTools}}.
\section{Entradas e configurações}
Iniciamos com limpeza das variáveis no \textit{workspace} e da tela. Também há configurações do script.
\begin{minted}[mathescape, style = autumn,
    frame = single, fontsize=\footnotesize]{matlab}
clear all;
clc;
tic; % iniciando contagem de tempo computacional
%   Arquivo de dados da rede
Rede_4_barras_valida
%   Conversão graus <-> radianos
graus_to_rad = pi/180;
rad_to_graus = 180/pi;
%pneves: Enable debug prints.
%pneves: Better to use breakpoints.Used here to didactic purposes
Debug = true;
%	Determinacao do tamanho da rede
fprintf('\n> Processando dados da rede ...\n')
%pneves: nb and nr are varibles of interest.
%pneves: variable colunas won't be used. 
[nb , colunas] = size(barras);
[nr , colunas] = size(ramos);
\end{minted}
%%%%%%%%%%%%%%%%%%%%%%%%%%%%%%%%%%%%%%%%%%%%%%%%%%%%%%
Carrega-se então, os valores da rede nas variáveis apropriadas. Vetores receberão dados das barras.
\begin{minted}[mathescape, style = autumn,
    frame = single, fontsize=\footnotesize]{matlab}
%Número - Tipo - V - Ângulo - Pg - Qg - Pc - Qc - bshk
for k = 1:nb            %pneves: extracting bar values into arrays
    numext(k)   = barras(k,1);                  
    tipo(k)     = barras(k,2);                  
    v(k)        = barras(k,3);                  
    ang(k)      = barras(k,4) * graus_to_rad;   
    pg(k)       = barras(k,5)/baseMVA; %pneves: PU 
    qg(k)       = barras(k,6)/baseMVA;
    pc(k)       = barras(k,7)/baseMVA;
    qc(k)       = barras(k,8)/baseMVA;
    bshk(k)     = barras(k,9)/baseMVA;
        pnom(k)     = pg(k) - pc(k); %pneves: P liquido
        qnom(k)     = qg(k) - qc(k); %pneves: Q liquido
    numint(barras(k,1)) = k;
end
\end{minted}
%%%%%%%%%%%%%%%%%%%%%%%
Vetores receberão dados dos ramos.
%FALAR COMO É CALCULADO OS MISMATCHES
\begin{minted}[mathescape, style = autumn,
    frame = single, fontsize=\footnotesize]{matlab}
for l = 1:nr                %	Carregamento dos vetores de ramo
	de(l)   = ramos(l,1);
	para(l) = ramos(l,2);
	r(l)    = ramos(l,3);
	x(l)    = ramos(l,4);
	bshl(l) = ramos(l,5)/2.; %shunt aqui é dividido por 2
	tap(l)  = ramos(l,6);
	fi(l)   = 0; %- ramos(l,7) * graus_to_rad;
end
fprintf('\n> Modelo de carga: potência constante (default).\n')
\end{minted}
%%%%%%%%%%%%%%%%%%%%%%%%%%%%%%%%%%%%%%%%%%%%%%%%
Neste ponto do código, será criada a matriz de admitância. A regra de formação dessa matriz está descrita em \ref{AdmitanciaElementosForaDiagonal} e \ref{AdmitanciaElementosDiagonal}. 
\begin{minted}[mathescape, style = autumn,
    frame = single, fontsize=\footnotesize]{matlab}
%pneves: Matriz Y dimensionada para NBxNB
Y = zeros(nb,nb);
for k = 1:nb
    Y(k,k) = i*bshk(k); %pneves: Cada Y(k,k) recebe 0+i*shunt
end
\end{minted}
%%%%%%%%%%%%%%%%%%%%%%%%%%%%%%%%%%%%%%%%%%%%%%
Preenchimento da matriz admitância.
\begin{minted}[mathescape, style = autumn,
    frame = single, fontsize=\footnotesize]{matlab}
for l = 1:nr %pneves: As formulas de formação da matriz
    k = de(l)%estão no Tópico Matriz de Admitância  
    m = para(l)
    %pneves: resistência será desprezada nesse problema
    %y(l) = 1/(r(l) + i*x(l)) 
    y(l) = 1/(-i*x(l)) 
    Y(k,k) = Y(k,k) + y(l)
    Y(m,m) = Y(m,m) + y(l)
    Y(k,m) = Y(k,m) - y(l)
    Y(m,k) = Y(m,k) - y(l)
end
G = real(Y); %pneves: matriz de condutância
B = imag(Y); %pneves: matriz de susceptância
if (Debug == true) %pneves: Debug
    fprintf('DEBUG> Y, G and B complete\n');
    Y
    G
    B
end
\end{minted}
A matriz $G$ será composta por 0, já que a resistência foi desprezada e a matriz $B$ originará a matriz $B'$.
\begin{minted}[mathescape, style = autumn,
    frame = single, fontsize=\footnotesize]{matlab}
%%
%pneves: Montando a matriz B'
Blinha = B;
P=pnom;
for k = 1:nb
    if tipo(k) == 3
        %pneves: retira a linha da barra slack
        Blinha(k,:) = []
        %pneves: retira a coluna da barra slack
        Blinha(:,k) = []
        %pneves: retira a linha da barra slack
        P(k,:) = []
    end
end
\end{minted}
Matriz $\theta$ é calculada, como descrito em \ref{Pk_linearizado_InjecaoLiquida_Matricial_Transposta}. 
\begin{minted}[mathescape, style = autumn,
    frame = single, fontsize=\footnotesize]{matlab}
%pneves: Calculando a matriz Teta
Theta = inv(Blinha)*P
%pneves: Adicionando a barra de referência na matriz Theta
IndexTheta=1;
for k = 1:nb
    if tipo(k) == 3
        ThetaWithReferenceBar(k,1) = 0;
    else
        ThetaWithReferenceBar(k,1) = Theta(IndexTheta,1);
        IndexTheta = IndexTheta + 1;
    end
end
%pneves: Convertendo de radianos para graus
ThetaGraus=Theta*rad_to_graus
\end{minted}
Por fim, os fluxos de potência são calculados, como em \ref{FigFluxoPotenciaLinearizado}.
\begin{minted}[mathescape, style = autumn,
    frame = single, fontsize=\footnotesize]{matlab}
%%
%'Fim do cálculo de fluxo de carga Linearizado');
 
fprintf('\n> Fim do cálculo de fluxo de carga.
                Preparando relatório de saída ...\n')
for l = 1:nr
    k = de(l);
    m = para(l);
    pkm(l) = (ThetaWithReferenceBar(k)- ThetaWithReferenceBar(m))/x(l);
end
%	Relatório final
fprintf('\n\n> Relatório final\n\n')
fprintf('> Rede: %s\n',nome_da_rede)
fprintf('\n> Fluxos de potência\n\n')
fprintf('     De Para       Pkm \n')

for l = 1:nr
	fprintf('%7d %4d %9.4f \n',de(l),para(l),pkm(l))
end

\end{minted}