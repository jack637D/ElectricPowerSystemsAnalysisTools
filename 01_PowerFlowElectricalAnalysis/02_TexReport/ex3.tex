\chapter{C\'odigo comentado}
\label{SectionCodigo}

Os códigos fonte desse trabalho bem como histórico de modificação, podem ser encontrados no endereço: \textit{\href{https://github.com/phneves/ElectricPowerSystemsAnalysisTools}{https://github.com/phneves/ElectricPowerSystemsAnalysisTools}}.
\section{Entradas e configurações}
Iniciamos com limpeza das variáveis no \textit{workspace} e da tela. Também há configurações do script.
\begin{minted}[mathescape, style = autumn,
    frame = single, fontsize=\footnotesize]{matlab}
clear all;
clc;
tol = 0.00001; 
% Arquivo de dados da rede
%P2GTD
RedePequena
graf = 'nao'; % Mostrar graficos ao final do cálculo
itmax = 20;% Número máximo de iterações permitido
% Tensões mínima e máxima permitidas
vmin = 0.0;
vmax = 2.0;
% Considerar cargas dependentes da tensão?
% 'nao'->modelo de pot cte. 'sim'->ModeloCarga deve ser especificado
cargasV = 'nao';
graus_to_rad = pi/180;% Conversão graus <-> radianos
rad_to_graus = 180/pi;
Debug = true; %pneves: Enable debug prints.
\end{minted}
%%%%%%%%%%%%%%%%%%%%%%%%%%%%%%%%%%%%%%%%%%%%%%%%%%%%%%
Carrega-se então, os valores da rede nas variáveis apropriadas. Vetores receberão dados das barras.
\begin{minted}[mathescape, style = autumn,
    frame = single, fontsize=\footnotesize]{matlab}
%Número - Tipo - V - Ângulo - Pg - Qg - Pc - Qc - bshk
for k = 1:nb            %pneves: extracting bar values into arrays
    numext(k)   = barras(k,1);                  
    tipo(k)     = barras(k,2);                  
    v(k)        = barras(k,3);                  
    ang(k)      = barras(k,4) * graus_to_rad;   
    pg(k)       = barras(k,5)/baseMVA; %pneves: PU 
    qg(k)       = barras(k,6)/baseMVA;
    pc(k)       = barras(k,7)/baseMVA;
    qc(k)       = barras(k,8)/baseMVA;
    bshk(k)     = barras(k,9)/baseMVA;
        pnom(k)     = pg(k) - pc(k); %pneves: P liquido
        qnom(k)     = qg(k) - qc(k); %pneves: Q liquido
    numint(barras(k,1)) = k;
end
\end{minted}
%%%%%%%%%%%%%%%%%%%%%%%
Vetores receberão dados dos ramos.
%FALAR COMO É CALCULADO OS MISMATCHES
\begin{minted}[mathescape, style = autumn,
    frame = single, fontsize=\footnotesize]{matlab}
for l = 1:nr                %	Carregamento dos vetores de ramo
	de(l)   = ramos(l,1);
	para(l) = ramos(l,2);
	r(l)    = ramos(l,3);
	x(l)    = ramos(l,4);
	bshl(l) = ramos(l,5)/2.; %shunt aqui é dividido por 2
	tap(l)  = ramos(l,6);
	fi(l)   = 0; %- ramos(l,7) * graus_to_rad;
end
fprintf('\n> Modelo de carga: potência constante (default).\n')
\end{minted}
%%%%%%%%%%%%%%%%%%%%%%%%%%%%%%%%%%%%%%%%%%%%%%%%
Neste ponto do código, será criada a matriz de admitância. A regra de formação dessa matriz está descrita em \ref{AdmitanciaElementosForaDiagonal} e \ref{AdmitanciaElementosDiagonal}. 
\begin{minted}[mathescape, style = autumn,
    frame = single, fontsize=\footnotesize]{matlab}
%pneves: Matriz Y dimensionada para NBxNB
Y = zeros(nb,nb);
for k = 1:nb
    Y(k,k) = i*bshk(k); %pneves: Cada Y(k,k) recebe 0+i*shunt
end
\end{minted}
%%%%%%%%%%%%%%%%%%%%%%%%%%%%%%%%%%%%%%%%%%%%%%
Preenchimento da matriz admitância.
\begin{minted}[mathescape, style = autumn,
    frame = single, fontsize=\footnotesize]{matlab}
for l = 1:nr %pneves: As formulas de formação da matriz
    k = de(l)%estão no Tópico Matriz de Admitância  
    m = para(l)
    y(l) = 1/(r(l) + i*x(l))
    akk(l) = 1/(tap(l)*tap(l))
    amm(l) = 1.0
    akm(l) = 1/tap(l)
    Y(k,k) = Y(k,k) + akk(l)*y(l) + i*bshl(l)
    Y(m,m) = Y(m,m) + amm(l)*y(l) + i*bshl(l)
    Y(k,m) = Y(k,m) - akm(l)*y(l)
    Y(m,k) = Y(m,k) - akm(l)*y(l)
end
G = real(Y); %pneves: matriz de condutância 
B = imag(Y); %pneves: matriz de susceptância
if (Debug == true) %pneves: Debug
    fprintf('DEBUG> Y, G and B complete\n');
    Y
    G
    B
end
\end{minted}
Aqui será definido o estado inicial da rede, com $V = 1$ PU para barras $PQ$, com $\theta = 0$ para barras $PQ$ e $PV$. O chamado \textit{Flat guess}.
\begin{minted}[mathescape, style = autumn,
    frame = single, fontsize=\footnotesize]{matlab}
for k = 1:nb
    if tipo(k) ~= 3
        ang(k) = 0.0;
        if tipo(k) < 2
            v(k)= 1.0;
        end
    end
end
\end{minted}

\section{Subsistema 1 - Iterativo}
Calcula-se as Potências nodais.\\
$pcalc(k)$ e $qcalc(k)$ são calculados para cada barra. Eles serão usados nos calculos de $H_{kk}$, $N_{kk}$, $M_{kk}$ e $L_{kk}$, submatrizes da matriz Jacobiana \ref{Jacobiana}.\\
\begin{minted}[mathescape, style = autumn,
frame = single, fontsize=\footnotesize]{matlab}
iter = 0;       %	Inicializar contador de iterações
maxDP = 10^5;   %	Inicializar maiores mismatches
maxDQ = 10^5;   
while abs(maxDP) > tol | abs(maxDQ) > tol %	Processo iterativo 
    %Cálculo das potências nodais
    for k = 1:nb
        pcalc(k) =  G(k,k)*v(k)*v(k);
        qcalc(k) = -B(k,k)*v(k)*v(k);
    end
    for l = 1:nr
        k = de(l);
        m = para(l);
        ab = ang(k) - ang(m) + fi(l);
        gkm = akm(l)*real(y(l));
        bkm = akm(l)*imag(y(l));
        pcalc(k) = pcalc(k) + v(k)*v(m)*(-gkm*cos(ab)-bkm*sin(ab));
        pcalc(m) = pcalc(m) + v(k)*v(m)*(-gkm*cos(ab)+bkm*sin(ab));
        qcalc(k) = qcalc(k) + v(k)*v(m)*(-gkm*sin(ab)+bkm*cos(ab));
        qcalc(m) = qcalc(m) - v(k)*v(m)*(-gkm*sin(ab)-bkm*cos(ab));
    end
\end{minted}
%%%%%%%%%%%%%%%%%%%%%%%
%FALAR DE ONDE VEM OS PCALC, QCALC. QUAL EQUAÇÃO AQUI ELES SE RELACIONAM
%FALAR COMO É CALCULADO OS MISMATCHES
As equações de $pcalc(k)$ (bloco de código acima) são termos somatórios descritas nas equações \ref{H_resolvido}, \ref{N_resolvido}, \ref{M_resolvido} e \ref{L_resolvido}.\\
No bloco de código abaixo, tem-se o cáculo dos \textit{mismatches} de potência (P e Q).
\begin{equation}
\left\{    \begin{array}{lll}
                DP(k)=pesp(k)-pcalc(k)\\
                DQ(k)=qesp(k)-qcalc(k)
            \end{array}\right.
    \label{DePcalc}
\end{equation}
As equações descritas em \ref{DePcalc} são as diferenças entre as potências esperadas e as calculadas, respectivamente para P e Q.\\
Esses valores serão os novos $\Delta P$ e $\Delta Q$, como descrito na equação \ref{delta_x_nu}.\\
No bloco subsequente, a variável de controle do número de iteração é incrementada. Neste algorítmo, $\nu$ está descrito como \textit{iter}.\\
O vetor $DS$ é montado. 
%\newpage
\begin{minted}[mathescape, style = autumn,
    frame = single, fontsize=\footnotesize]{matlab}
    %Cálculo dos mismatches de potência
    DP = zeros(nb,1);
    DQ = zeros(nb,1);
    maxDP = 0;
    maxDQ = 0;
    busDP = 0;
    busDQ = 0;
    for k = 1:nb
        pesp(k) = pnom(k);
        qesp(k) = qnom(k);
        if tipo(k) ~= 3
            DP(k) = pesp(k) - pcalc(k);
            if abs(DP(k)) > abs(maxDP)
                maxDP = DP(k);
                busDP = numext(k);
            end
        end
        if tipo(k) <= 1
            DQ(k) = qesp(k) - qcalc(k);
            if abs(DQ(k)) > abs(maxDQ)
                maxDQ = DQ(k);
                busDQ = numext(k);
            end
        end
    end
\end{minted}
%%%%%%%%%%%%%%%%%%%%%%%%%%%%%%%%%%%%%%%%%%%%%%%%%%%%
\begin{minted}[mathescape, style = autumn,
    frame = single, fontsize=\footnotesize]{matlab}
    iteracao(iter+1) = iter;
    mismP(iter+1) = abs(maxDP);
    mismQ(iter+1) = abs(maxDQ);

    DS = [ DP ; DQ ];

    fprintf('\n> Iteração - %d\n',iter)
    fprintf('* Máximo mismatch P = %06.4f (barra %04d)\n',maxDP,busDP)
    fprintf('* Máximo mismatch Q = %06.4f (barra %04d)\n',maxDQ,busDQ)
\end{minted}
Caso o \textit{mismatch} seja maior que a tolerância, a matriz Jacobiana será montada e o estado será atualizado.
\begin{minted}[mathescape, style = autumn,
    frame = single, fontsize=\footnotesize]{matlab}
    if abs(maxDP) > tol | abs(maxDQ) > tol
        %	Montar e inverter a matriz Jacobiana
        H = zeros(nb,nb); M=H; N=H; L=H;
        for k = 1:nb
            H(k,k) = -qcalc(k) - v(k)*v(k)*B(k,k);
            if tipo(k) == 3
                H(k,k) = 10^10;
            end
            N(k,k) = ( pcalc(k) + v(k)*v(k)*G(k,k) )/v(k) ;
            M(k,k) = pcalc(k) - v(k)*v(k)*G(k,k);
            L(k,k) = ( qcalc(k) - v(k)*v(k)*B(k,k) )/v(k) ;
            if tipo(k) >= 2
                L(k,k) = 10^10;
            end
        end
        for l = 1 : nr,
              k = de(l) ;
              m = para(l) ;
              ab = ang(k) - ang(m) + fi(l) ;
                
              H(k,m) =   v(k)*v(m)*( G(k,m)*sin(ab)-B(k,m)*cos(ab)) ;
              H(m,k) =   v(k)*v(m)*(-G(k,m)*sin(ab)-B(k,m)*cos(ab)) ;
              N(k,m) =   v(k)*(G(k,m)*cos(ab)+B(k,m)*sin(ab)) ;
              N(m,k) =   v(m)*(G(k,m)*cos(ab)-B(k,m)*sin(ab)) ;
              M(k,m) = - v(k)*v(m)*(G(k,m)*cos(ab)+B(k,m)*sin(ab)) ;
              M(m,k) = - v(k)*v(m)*(G(k,m)*cos(ab)-B(k,m)*sin(ab)) ;
              L(k,m) =   v(k)*(G(k,m)*sin(ab)-B(k,m)*cos(ab)) ;
              L(m,k) = - v(m)*(G(k,m)*sin(ab)+B(k,m)*cos(ab)) ;
        end
        J = [ H N ; M L ];
        %	Calcular o vetor de correção de estado
        DV = inv(J)*DS;
        %	Atualizar estado
        v = v + DV(nb+1:2*nb)';
        ang = ang + DV(1:nb)';
        
\end{minted}
%%%%%%%%%%%%%%%%%%%%%%%
Verifica-se se há divergência, ou seja, se o método está caminhando para uma resposta. 
\begin{minted}[mathescape, style = autumn,
    frame = single, fontsize=\footnotesize]{matlab}
        %	Verificar se houve divergência
        for k = 1:nb
            if v(k) < vmin | v(k) > vmax
                fprintf('\n> Tensões fora dos limites ->
                divergência.\n')
                fprintf('> Execução interrompida.\n\n')
                if graf == 'sim'
                            graficos
                end
                msgbox('Tensões fora dos limites -> divergência.
                Execução interrompida.','Aviso','warn');
                return
            end
        end
\end{minted}
%%%%%%%%%%%%%%%%%%%%%%%
Verifica-se se o método excedeu o número de iterações máximas.
\begin{minted}[mathescape, style = autumn,
    frame = single, fontsize=\footnotesize]{matlab}
    %	Incrementar contador de iterações

    iter = iter + 1;

        if iter > itmax
            fprintf('\n> Número máximo de iterações permitido foi
            excedido.\n')
            fprintf('> Execução interrompida.\n\n')
            if graf == 'sim'
                graficos
            end
            msgbox('Número máximo de iterações permitido foi
            excedido. Execução interrompida.','Aviso','warn');
            return
        end
    end
end	% final do while
\end{minted}
\section{Subsistema 2 - Cálculo de $P_k$ e $Q_k$}
Neste ponto, o software já fez todos os calculos e se saiu do laço de repetição \textit{while}, então ele convergiu para resposta.\\
As potências serão recalculadas com os valores finais para gerar o relatório.
\begin{minted}[mathescape, style = autumn,
    frame = single, fontsize=\footnotesize]{matlab}
%%
%'Fim do cálculo de fluxo de carga. Preparando relatório de saída');
fprintf('\n> Fim do cálculo de fluxo de carga. 
Preparando relatório de saída ...\n')
\end{minted}
Será calculada as potências nodais.
\begin{minted}[mathescape, style = autumn,
    frame = single, fontsize=\footnotesize]{matlab}
%	Cálculo das potências nodais
for k = 1:nb
	pcalc(k) = G(k,k)*v(k)*v(k);
    qcalc(k) = -B(k,k)*v(k)*v(k);
end
for l = 1:nr
	k = de(l);
    m = para(l);
    ab = ang(k) - ang(m) + fi(l);
    gkm = akm(l)*real(y(l));
    bkm = akm(l)*imag(y(l));
    pcalc(k) = pcalc(k) + v(k)*v(m)*(-gkm*cos(ab)-bkm*sin(ab));
    pcalc(m) = pcalc(m) + v(k)*v(m)*(-gkm*cos(ab)+bkm*sin(ab));
    qcalc(k) = qcalc(k) + v(k)*v(m)*(-gkm*sin(ab)+bkm*cos(ab));
    qcalc(m) = qcalc(m) - v(k)*v(m)*(-gkm*sin(ab)-bkm*cos(ab));
end
\end{minted}
Será calculado o fluxo de potência nos ramos
\begin{minted}[mathescape, style = autumn,
    frame = single, fontsize=\footnotesize]{matlab}
%	Cálculo dos fluxos de potência nos ramos
for l = 1:nr
    k = de(l);
    m = para(l);
    gkm = real(y(l));
    bkm = imag(y(l));
    ab = ang(k) - ang(m) + fi(l);
    vkm = v(k)*v(m);
    pkm(l) = akk(l)*v(k)*v(k)*gkm -
        akm(l)*vkm*(gkm*cos(ab)+bkm*sin(ab));
    pmk(l) = amm(l)*v(m)*v(m)*gkm -
        akm(l)*vkm*(gkm*cos(ab)-bkm*sin(ab));
    qkm(l) = -akk(l)*v(k)*v(k)*(bkm+bshl(l)) +
        akm(l)*vkm*(bkm*cos(ab)-gkm*sin(ab));
    qmk(l) = -amm(l)*v(m)*v(m)*(bkm+bshl(l)) +
        akm(l)*vkm*(bkm*cos(ab)+gkm*sin(ab));
    pperdas(l) = pkm(l) + pmk(l);
    qperdas(l) = qkm(l) + qmk(l);
end
\end{minted}
Apresentação do relatório final. 
\begin{minted}[mathescape, style = autumn,
    frame = single, fontsize=\footnotesize]{matlab}
fprintf('\n\n> Relatório final\n\n') %	Relatório final
fprintf('  Convergiu em %d iterações\n\n',iter)
fprintf('> Estado da rede\n\n')
fprintf('   Barra Tipo   Mag Fase    P   Q   Qsh\n')
for k = 1:nb
    fprintf('%7d %4d %9.4f %6.2f %9.4f %7.4f %9.4f \n',
    numext(k),tipo(k),v(k),(ang(k)*rad_to_graus),pcalc(k),
    qcalc(k), bshk(k)*v(k)^2)
end
fprintf('\n> Fluxos de potência\n\n')
fprintf('   De  Para    Pkm   Qkm   Pmk   Qmk Ploss   Qloss\n')
for l = 1:nr
    fprintf('%7d %4d %9.4f %7.4f %9.4f %7.4f %9.4f %7.4f\n',
    de(l),para(l),pkm(l),qkm(l),pmk(l),qmk(l),pperdas(l),qperdas(l))
end
if graf == 'sim'
    graficos
end
tempo = toc; % terminando contagem de tempo computacional
fprintf('\n\n> Tempo computacional = %7.4f segundos.',tempo)
fprintf('\n\n> Fim da simulação.\n\n')
\end{minted}
